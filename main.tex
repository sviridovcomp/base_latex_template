\documentclass[a4paper, 12pt]{article} % Here we specify the paper size, font size and document type

\usepackage{cmap} % Make pdf searchable
\usepackage[T2A]{fontenc} % 
\usepackage[utf8]{inputenc} % encoding on source document
\usepackage[english, russian]{babel} % Multi language supporting
\usepackage{graphicx} % This package allows working with images
\usepackage{mathtools} % This package need to working with math

\usepackage{indentfirst}

\begin{document}

\LaTeX \  -- это своего рода препроцессор текста для \TeX \ -- программы компьютерной верстки. \LaTeX \ является программируемым и расширяемым, что позволяет автоматизировать большую часть аспектов набора, включая нумерацию, перекрёстные ссылки, таблицы и изображения (их размещение и подписи к ним), общий вид страницы, библиографию и многое-многое другое. \LaTeX \ был первоначально написан Лэсли Лампортом в 1984-м году и стал наиболее популярным способом использования \TeX а; очень мало людей сегодня пишут на оригинальном \TeX e. Текущей версией является \LaTeX \ 2$\varepsilon$.
2 формулы


\end{document}